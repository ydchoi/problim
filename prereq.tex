\documentclass{article} % For LaTeX2e
\usepackage{nips14submit_e,times}
\usepackage{amsmath}
\usepackage{amsthm}
\usepackage{amssymb}
\usepackage{mathtools}
\usepackage{hyperref}
\usepackage{url}
\usepackage{algorithm}
\usepackage[noend]{algpseudocode}
%\documentstyle[nips14submit_09,times,art10]{article} % For LaTeX 2.09

\usepackage{graphicx}
\usepackage{caption}
\usepackage{subcaption}

\def\eQb#1\eQe{\begin{eqnarray*}#1\end{eqnarray*}}
\def\eQnb#1\eQne{\begin{eqnarray}#1\end{eqnarray}}
\providecommand{\e}[1]{\ensuremath{\times 10^{#1}}}
\providecommand{\pb}[0]{\pagebreak}


\def\Qb#1\Qe{\begin{question}#1\end{question}}
\def\Sb#1\Se{\begin{solution}#1\end{solution}}

\newenvironment{claim}[1]{\par\noindent\underline{Claim:}\space#1}{}
\newtheoremstyle{quest}{\topsep}{\topsep}{}{}{\bfseries}{}{ }{\thmname{#1}\thmnote{ #3}.}
\theoremstyle{quest}
\newtheorem*{definition}{Definition}
\newtheorem*{theorem}{Theorem}
\newtheorem*{lemma}{Lemma}
\newtheorem*{question}{Question}
\newtheorem*{preposition}{Preposition}
\newtheorem*{exercise}{Exercise}
\newtheorem*{challengeproblem}{Challenge Problem}
\newtheorem*{solution}{Solution}
\newtheorem*{remark}{Remark}
\usepackage{verbatimbox}
\usepackage{listings}
\title{Self-test Questions on Prerequisites}


\author{
Youngduck Choi \\
Courant Institute of Mathematical Sciences \\
New York University \\
\texttt{yc1104@nyu.edu} \\
}


% The \author macro works with any number of authors. There are two commands
% used to separate the names and addresses of multiple authors: \And and \AND.
%
% Using \And between authors leaves it to \LaTeX{} to determine where to break
% the lines. Using \AND forces a linebreak at that point. So, if \LaTeX{}
% puts 3 of 4 authors names on the first line, and the last on the second
% line, try using \AND instead of \And before the third author name.

\newcommand{\fix}{\marginpar{FIX}}
\newcommand{\new}{\marginpar{NEW}}

\nipsfinalcopy % Uncomment for camera-ready version

\begin{document}


\maketitle

\begin{abstract}
The following is a collection of solutions of the self-test questions for Real
Variables at the Courant Institute.
\end{abstract}

\begin{question}
\end{question}
\begin{solution}
Two sets are said to be equipotent provided there is a bijective map from one to another.
Hence, to show that the sets $(0,1]$ and $[0,1]$ are equipotent, it suffices to construct
a bijective map from $(0,1]$ to $[0,1]$. 
\end{solution}

\bigskip

\begin{question}[1-2. Equipotence is an RST relation]
\end{question}
\begin{solution}
We prove that equipotence is an equivalence relation on sets, denoted as $R$.
First, a set is equipotent with
itself, as the identity map establishes an equipotence. Second, let $(A,B) \in R$. Then,
by the definition of equipotence, there exists a map $f:A \to B$ such that $f$ is a one-to-one
correpondence. Now, the inverse relation of the map $f$, $f^{-1}$, is also a one-to-one map from
$B$ to $A$. Hence, $(B,A) \in R$, and $R$ is reflexive. Now, let $(A,B)$ and $(B,C)$ be elements
in $R$. Then, there exists two bijective maps $f_{AB}$ and $f_{BC}$. Consider the composition of
the two maps $f_{AC}:A \to C$. The map $f_{AC}$ is a bijective map from $A$ to $C$. Hence,
there exists a one-to-one correspondence between $A$ and $C$. Hence, $R$ is transitive. Therefore,
$R$ is an equivalence relation. $\qed$
\end{solution}

\begin{question}[1-3]
\end{question}
\begin{solution}
Let $E$ be a nonempty subset of the real numbers. We want to show that $\inf E = \sup E$ iff $E$ contains
a single point. Assume that $E$ is a single point, thus $E = \{ x \}$. As, $x \geq x$ and $x \leq x$, $x$
is both $\sup E$ and $\inf E$. Hence, $\inf E = \sup E$. Assume that $\inf E = \sup E$. By the definition of
supremum and infimum, we have that for all $x \in E$, we have $\inf E \leq x \leq \sup E$. Combined with $\inf E
= \sup E$, we have $\inf E = x = \sup E$. Hence, $E$ is a single point set.
\end{solution}

\bigskip

\begin{question}[The Cauchy Convergence Criterion for Real Sequences]
\end{question}
\begin{solution}
Let $\{ a_n \}$ be a sequence of real numbers. First, assume that $\{ a_n \} \to a$.
Then, for all natural numbers $n$ and $m$, by the triangle inequality, we have
\eQb
|a_n - a_m| = |(a_n -a) - (a_m -a)| \leq |a_n - a| - |a_m - a|.
\eQe
As $\{ a_n \}$ is convergent, for any $\epsilon > 0$, we have $N$ such that
$|a_k - a| < \dfrac{\epsilon}{2}$, for $k \geq N$. Hence, there exists $N$, such that
for $n,m \geq N$, we have $N$ such that $|a_n - a| < \dfrac{\epsilon}{2}$ 
and $|a_m - a| < \dfrac{\epsilon}{2}$, thus $|a_n - a_m| < \dfrac{\epsilon}{2}$. $\{ a_n \}$
is cauchy.
\end{solution}

\pagebreak

\begin{question}[1-4. $\sigma$-algebra]
\end{question}
\begin{solution}
Let $F$ be a collection of subsets of $X$, and let
$\{ A_{\lambda} \}_{\lambda \in \Lambda}$ be 
a collection of collections of subsets of $X$ that contains
$F$. Consider $\cap_{\lambda \in \Lambda} A_{\lambda}$. Clearly, $F \in 
\cap_{\lambda \in \Lambda} A_{\lambda}$. We now want to show that $\cap_{\lambda \in \Lambda}
A_{\lambda}$ is indeed a $\sigma$-algebra. $\emptyset$ and $X$ are in 
$\cap_{\lambda \in \Lambda}$, as they are in every $\sigma$-algebra. It remains to show that
it is "closed" under countable union and complement. Let $E \in \cap_{\lambda \in \Lambda} A_{\lambda}$.
Then, $E$ is in $A_{\lambda}$ for all $\lambda \in \Lambda$. As each $A_{\lambda}$s are $\sigma$-algebra,
$E^{C}$ is in $A_{\lambda}$ for all $\lambda \in \Lambda$.

\end{solution}

\bigskip

\begin{question}[1.5. Further subsequence]
\end{question}
\begin{solution}
Suppose for sake of contradiction that $\{ x_n \}$ does not converge to $x$. Then, for some $\epsilon > 0$,
for all $N \in \mathbb{N}$, there exists $x_n$ with $n \geq N$, such that $|x_n - x| \geq \epsilon$. Then,
for each $N \in \mathbb{N}$, pick an element that satisfies $|x_N - x| \geq \epsilon$; this is a 
subsequence of $\{ x_n \}$, which we denote as $\{ x_{n_k} \}$. We now have that 
$|x_{n_k} - x| \geq \epsilon $ for all $k$. Hence, this particular subsequence of $\{ x_n \}$ cannot
have a further subsequence that converges to $x$, which is a contradiction. Hence, $\{ x_n \} \to x$.
$\qed$
\end{solution}

\bigskip

\begin{question}[1.4. $\limsup$ is $\sup C$]
\end{question}
\begin{solution}
Let $\{ a_n \}$ be a sequence of
real numbers, $C$ be a set of cluster points of $\{ a_n \}$. 
First, 
we simply denote $\limsup \{ a_n \}$ as $s$, which can be written as
\[
s = 
\underset{n \to \infty}{\lim} [ \> \sup \{ a_k \>\> | \>\> k \geq n \} \> ].
\]
We first show that $s \in C$. We have two cases. First, assume $|s| = \infty$. Then, the sequence 
is divergent, and any subsequence of the sequence converges to either $\infty$ or $-\infty$, 
depending on the situation. Hence, $s$ is a cluster point. Second, assume that $|s| < \infty$.
We would like to construct a subsequence $\{a_{n_k} \}$, which converges to $s$. 
Let $s_n = \sup \{ a_k \>\> | \>\> k \geq n \}$. By the approximation property of supremum, 
there exists an index $n_1$, such that $s_1 - \dfrac{1}{2} < a_{n_1} < s_1$ holds.
Now, consider an inductive selection process, where given the choice of $a_{n_k}$, by using
the approximation property of supremum, we pick
$a_{n_{k+1}}$ such that 
\[
s_{a_{n_k}+1} - \dfrac{1}{2^{a_{n_k}+1}} < a_{n_{k+1}} < s_{a_{n_k}+1},
\]
holds. Now, the sequences on the right hand side and the left hand side both converge to $s$,
and by squeeze theorem, the constructed sequence $\{ a_{n_k} \}$ converges to $s$. Hence,
$\limsup \{ a_n \}$ is a cluster point.
\\
\\
Now, we show that $\limsup \{ a_n \}$ is the
largest cluster point. When $\limsup \{ a_n \}$ is unbounded, it trivially holds. Assume that
$\limsup \{ a_n \}$ is a real number.
Let $x$ be any cluster point of $\{ a_n \}$. By the definition of a cluster point,
we have a subsequence $\{ a_{n_k} \} $ such that converges to $x$. Notice that, by the definition of 
$\limsup$, we have $ a_{n_k} \leq s_{n_k}$ for all $k$, as $a_{n_k}$ is an element of the set considered
for the $s_{n_k}$ term. As $\{ s_{n_k} \}$ is a subsequence of $\{s_n \}$, it also converges to $s$ and
hence, $s \geq x$. We have
shown that $\limsup \{ a_n \}$ is the largest cluster point.
\end{solution}

\bigskip

\begin{question}[$\limsup$]
\end{question}
\begin{solution}
Let $\{ a_n \}$, and $\{ b_n \}$ be real valued sequences.
\end{solution}

\bigskip

\begin{question}[2. Continous functions]
\end{question}
\begin{solution}
Let $f:\mathbb{R} \to \mathbb{R}$ be a continuous function, and assume that $f(0) > 0$. By the $\epsilon-\delta$
criterion of continuity at $0$, we have that for any $\epsilon > 0$, there exists $\delta > 0$ such that for $x \in 
\mathbb{R}$, if $|x - 0| < \delta$, then $|f(x) - f(0)| < \epsilon$. Set $\epsilon = \dfrac{f(0)}{2}$. Then, 
we have there exists $\delta > 0$ such that for $x \in B(0,\delta)$, $|f(x) - f(0)| < \dfrac{f(0)}{2}$, thus
$f(x) > 0$. Hence, we have shown that there exists a nonempty interval $(\delta, \delta)$, where $\delta$ is chosen
from the continuity criterion with respect to $\dfrac{f(0)}{2}$, that all elements inside is strictly positive.
$\qed$
\end{solution}

\begin{question}
\end{question}
\begin{solution}
Let $f:\mathbb{R} \to \mathbb{R}$ be a function. Assume that $f$ is continuous.  
\end{solution}

\end{document}


















