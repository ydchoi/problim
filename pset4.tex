\documentclass{article} % For LaTeX2e
\usepackage{nips14submit_e,times}
\usepackage{amsmath}
\usepackage{amsthm}
\usepackage{amssymb}
\usepackage{mathtools}
\usepackage{hyperref}
\usepackage{url}
\usepackage{mathrsfs}
\usepackage{algorithm}
\usepackage[noend]{algpseudocode}
%\documentstyle[nips14submit_09,times,art10]{article} % For LaTeX 2.09

\usepackage{graphicx}
\usepackage{caption}
\usepackage{subcaption}

\def\eQb#1\eQe{\begin{eqnarray*}#1\end{eqnarray*}}
\def\eQnb#1\eQne{\begin{eqnarray}#1\end{eqnarray}}
\providecommand{\e}[1]{\ensuremath{\times 10^{#1}}}
\providecommand{\pb}[0]{\pagebreak}

\usepackage{mathtools}
\DeclarePairedDelimiter{\ceil}{\lceil}{\rceil}
\DeclarePairedDelimiter{\floor}{\lfloor}{\rfloor}

\newcommand{\E}{\mathrm{E}}
\newcommand{\Var}{\mathrm{Var}}
\newcommand{\Cov}{\mathrm{Cov}}

\def\Qb#1\Qe{\begin{question}#1\end{question}}
\def\Sb#1\Se{\begin{solution}#1\end{solution}}

\newenvironment{claim}[1]{\par\noindent\underline{Claim:}\space#1}{}
\newtheoremstyle{quest}{\topsep}{\topsep}{}{}{\bfseries}{}{ }{\thmname{#1}\thmnote{ #3}.}
\theoremstyle{quest}
\newtheorem*{definition}{Definition}
\newtheorem*{theorem}{Theorem}
\newtheorem*{lemma}{Lemma}
\newtheorem*{question}{Question}
\newtheorem*{preposition}{Preposition}
\newtheorem*{exercise}{Exercise}
\newtheorem*{challengeproblem}{Challenge Problem}
\newtheorem*{solution}{Solution}
\newtheorem*{remark}{Remark}
\usepackage{verbatimbox}
\usepackage{listings}
\title{Basic Probability: \\
Problem Set IV}


\author{
Youngduck Choi \\
CILVR Lab \\
New York University\\
\texttt{yc1104@nyu.edu} \\
}


% The \author macro works with any number of authors. There are two commands
% used to separate the names and addresses of multiple authors: \And and \AND.
%
% Using \And between authors leaves it to \LaTeX{} to determine where to break
% the lines. Using \AND forces a linebreak at that point. So, if \LaTeX{}
% puts 3 of 4 authors names on the first line, and the last on the second
% line, try using \AND instead of \And before the third author name.

\newcommand{\fix}{\marginpar{FIX}}
\newcommand{\new}{\marginpar{NEW}}

\nipsfinalcopy % Uncomment for camera-ready version

\begin{document}


\maketitle

\begin{abstract}
This work contains a collection of solutions for selected problems 
of the Basic Probability course of Fall 2015.
\end{abstract}

\bigskip

\begin{question}[1]
\end{question}
\begin{solution}
We are given that
\eQb
P(M) = 0.5 \>\> &\text{ and }& \>\> P(F) = 0.5 \\
P(C|M) = 0.03 \>\> &\text{ and }& \>\> P(C|F) = 0.05, \\
\eQe
where $C$, $M$ and $F$ are events corresponding to a chosen person
being colorblind, male
and female respectively. Note that $M$ and $F$ form a partition
of the sample space. 
Now, by the Bayes' theorem, we have
\eQb
P(M|C) &=& \dfrac{P(C|M)P(M)}{P(C)}. \\
\eQe
As $M$ and $F$ is a partition of the sample space, we have
\eQb
P(M|C) &=& \dfrac{P(C|M)P(M)}{P(C|M)P(M) + P(F|M)P(M)}.
\eQe
Therefore, substituting the givens yields
\eQb
P(M|C) &=& \dfrac{0.03 \cdot 0.5}{0.03 \cdot 0.5 + 0.05 \cdot 0.5} \\
&=& 0.375.
\eQe
If there are twice many males over females, we have
\eQb
P(M|C) &=& \dfrac{0.03 \cdot \frac{2}{3}}{0.03 \cdot \frac{2}{3}
+ 0.05 \cdot \frac{1}{3}} \\
&\approx& 0.545.
\eQe
This completes the computation. $\qed$

\end{solution}

\pagebreak

\begin{question}[2]
\end{question}
\begin{solution}
We have an experiment of selecting a coin from a box of three coins at
random and flipping the selected coin. Let $C_1, C_2$ and $C_3$ be
events corresponding to a chosen coin being the coin $1$, coin $2$
and coin $3$. Let $H$ be an event that a head shows. Then,
we are given that
\eQb
P(C_1) = \dfrac{1}{3} \>\> &\text{ and }& P(H|C_1) = 1 \\
P(C_2) = \dfrac{1}{3} \>\> &\text{ and }& P(H|C_2) = 0.5 \\
P(C_3) = \dfrac{1}{3} \>\> &\text{ and }& P(H|C_3) = 0.65.
\eQe
By the Bayes' theorem, we have
\eQb
P(C_1|H) &=& \dfrac{P(H|C_1)P(C_1)}{P(H)}.
\eQe
As $C_1$, $C_2$ and $C_3$ form a partition of the sample space,
we have
\eQb
P(C_1|H) &=& \dfrac{P(H|C_1)P(C_1)}
{P(H|C_1)P(C_1) + P(H|C_2)P(C_2) + P(H|C_3)P(C_3)} \\
&=& \dfrac{1 \cdot \frac{1}{3}}{1 \cdot \frac{1}{3} + 
0.5 \cdot \frac{1}{3} + 0.65 \cdot \frac{1}{3}} \\
&\approx& 0.465.
\eQe
Therefore, the probability that it was the two-headed coin, given that
the throw resulted in a head is approximately $0.465$. $\qed$


\end{solution}

\bigskip

\begin{question}[3]
\end{question}
\begin{solution}
Given the distribution function $F$, by definition, we can write 
the probability of an event $(a,b)$ as
\eQb
P((a,b)) = F(b) - F(a).
\eQe
Therefore, it follows that
\eQb
P((-1/2,1/2)) &=& F(1/2) - F(-1/2) \\
&=& \dfrac{1}{4} - 0 \\
&=& \dfrac{1}{4} \\
P((1/2,3/2)) &=& F(3/2) - F(-1/2) \\
&=& \dfrac{3}{4} - 0 \\
&=& \dfrac{3}{4} \\
P((2/3,5/2)) &=& F(5/2) - F(2/3) \\
&=& 1 - \dfrac{1}{4} \\
&=& \dfrac{3}{4} \\
P((3,\infty)) &=& F(\infty) - F(3) \\ 
&=& 1 - 1 \\
&=& 0.
\eQe
This completes the computations. $\qed$
\end{solution}

\bigskip

\begin{question}[4]
\end{question}
\begin{solution}
We wish to compute the factorial moment of the geometric distribution.
Writing out definition of the factorial moment and simplifying with
geometric series and $r$th derivative, we have
\eQb
\mathbb{E}\left[ \dfrac{X!}{(X-r)!} \right]
&=& \sum_{k=r}^{\infty} \dfrac{k!}{(k-r)!}(1-p)^k p \\
&=& p(1-p)^r\sum_{k=r}^{\infty} \dfrac{k!}{(k-r)!}(1-p)^{k-r} \\
&=& p(1-p)^r (-1)^r \dfrac{d^r}{d_p^r} \sum_{k=0}^{\infty} 
(1-p)^k \\
&=& p(1-p)^r (-1)^r \dfrac{d^r}{d_p^r} \dfrac{1}{p} \\
&=& p(1-p)^r (-1)^r (-1)^r r! \dfrac{1}{p^{r+1}} \\
&=& \dfrac{r!(1-p)^r}{p^r}, 
\eQe
as desired. $\qed$
\end{solution}
\bigskip

\begin{question}[5]
\end{question}
\begin{solution}
Let $X$ have a binomial distribution with parameters $(p,n)$. 
Let $p \in (0,1)$. We proceed by mathematical induction. 
For the $n=1$, we have
\eQb
\dfrac{1}{2}(1+(1-2p)^1) &=& \dfrac{1}{2}(2-2p) \\
&=& (1-p) \\
&=& {1 \choose 0} p^0(1-p)^1
\eQe
Therefore the base case holds. Assume that the formula holds for $n$.
Let $X_n$ and $X_{n+1}$ be the binomial distribution with $n$ and $n+1$
parameters respectively.
Observe the following recurrence relation:
\eQb
P(X_{n+1}=even) = P(X_{n}=even)(1-p) + P(X_{n}=odd)p,
\eQe
which can be seen by partitioning the probability by the outcome of
the $n+1$ trial. If the last trial is a success, then the number of successes
upto $n$ must be odd. Similarly, if the last trial is a failure, then
the number of successes upto $n$ must be even. 
Substituting the inductive hypothesis into the above recurrence relation
yields
\eQb
&=& \dfrac{1}{2}(1+(1-2p)^n)(1-p) + (1-\dfrac{1}{2}(1+(1-2p)^n)p \\
&=& (\dfrac{1}{2} + \dfrac{1}{2}(1-2p)^n)(1-p) 
+ (\dfrac{1}{2} - \dfrac{1}{2}(1-2p)^n)p \\
&=& \dfrac{1}{2} + \dfrac{1}{2}(1-2p)^n(1-2p) \\
&=& \dfrac{1}{2}(1 + (1-2p)^{n+1}), \\
\eQe
which completes the induction. Therefore, we have shown that
$X$ is even with probability $\dfrac{1}{2}(1 + (1-2p)^n)$.
$\qed$
\end{solution}

\end{document}
