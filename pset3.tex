\documentclass{article} % For LaTeX2e
\usepackage{nips14submit_e,times}
\usepackage{amsmath}
\usepackage{amsthm}
\usepackage{amssymb}
\usepackage{mathtools}
\usepackage{hyperref}
\usepackage{url}
\usepackage{mathrsfs}
\usepackage{algorithm}
\usepackage[noend]{algpseudocode}
%\documentstyle[nips14submit_09,times,art10]{article} % For LaTeX 2.09

\usepackage{graphicx}
\usepackage{caption}
\usepackage{subcaption}

\def\eQb#1\eQe{\begin{eqnarray*}#1\end{eqnarray*}}
\def\eQnb#1\eQne{\begin{eqnarray}#1\end{eqnarray}}
\providecommand{\e}[1]{\ensuremath{\times 10^{#1}}}
\providecommand{\pb}[0]{\pagebreak}

\usepackage{mathtools}
\DeclarePairedDelimiter{\ceil}{\lceil}{\rceil}
\DeclarePairedDelimiter{\floor}{\lfloor}{\rfloor}

\newcommand{\E}{\mathrm{E}}
\newcommand{\Var}{\mathrm{Var}}
\newcommand{\Cov}{\mathrm{Cov}}

\def\Qb#1\Qe{\begin{question}#1\end{question}}
\def\Sb#1\Se{\begin{solution}#1\end{solution}}

\newenvironment{claim}[1]{\par\noindent\underline{Claim:}\space#1}{}
\newtheoremstyle{quest}{\topsep}{\topsep}{}{}{\bfseries}{}{ }{\thmname{#1}\thmnote{ #3}.}
\theoremstyle{quest}
\newtheorem*{definition}{Definition}
\newtheorem*{theorem}{Theorem}
\newtheorem*{lemma}{Lemma}
\newtheorem*{question}{Question}
\newtheorem*{preposition}{Preposition}
\newtheorem*{exercise}{Exercise}
\newtheorem*{challengeproblem}{Challenge Problem}
\newtheorem*{solution}{Solution}
\newtheorem*{remark}{Remark}
\usepackage{verbatimbox}
\usepackage{listings}
\title{Basic Probability: \\
Problem Set III}


\author{
Youngduck Choi \\
CILVR Lab \\
New York University\\
\texttt{yc1104@nyu.edu} \\
}


% The \author macro works with any number of authors. There are two commands
% used to separate the names and addresses of multiple authors: \And and \AND.
%
% Using \And between authors leaves it to \LaTeX{} to determine where to break
% the lines. Using \AND forces a linebreak at that point. So, if \LaTeX{}
% puts 3 of 4 authors names on the first line, and the last on the second
% line, try using \AND instead of \And before the third author name.

\newcommand{\fix}{\marginpar{FIX}}
\newcommand{\new}{\marginpar{NEW}}

\nipsfinalcopy % Uncomment for camera-ready version

\begin{document}


\maketitle

\begin{abstract}
This work contains a collection of solutions for selected problems 
of the Basic Probability course of Fall 2015.
\end{abstract}

\bigskip

\begin{question}[1. Limit and Conditional Probability]
\end{question}
\begin{solution}
We first claim that if $A_n \to A$ and $B_n \to B$, then $A_n \cap B_n \to A \cap B$.
Let $w \in \Omega$. As $A_n \to A$, there exists $N_A$ such that 
\eQb
1_{A_n}(w) &=& 1_{A}(w) \>\> \text{ for } \>\>  n \geq N_A. 
\eQe
Similarly, as $B_n \to B$, there exists $N_B$ such that 
\eQb
1_{B_n}(w) &=& 1_{B}(w) \>\> \text{ for } \>\>  n \geq N_B. 
\eQe
Take $N^* = \mathrm{max}(N_A,N_B)$. Then, for $n \geq N^*$, we have
\eQb
1_{A_n \cap B_n}(w) &=& 1_{A_n}(w)1_{B_n}(w) \\
&=& 1_{A}(w) 1_{B}(w) \\
&=& 1_{A \cap B}(w).
\eQe 
Since $w$ was arbitrary, we have shown that $A_n \cap B_n \to A \cap B$.

\smallskip

Recall the following theorem established earlier in the class:
let $P$ be a probability measure, and $A_n$ be a sequence of events in $\mathscr{A}$ which
converge to $A$. Then, $A \in \mathscr{A}$ and $\lim_{n \to \infty} \mathbb{P}(A_n) = 
\mathbb{P}(A)$. Combining the theorem with the above result we have that
if $A_n \to A$ and $B_n \to B$, we have
\eQb
\lim_{n \to \infty}\mathbb{P}(A_n \cap B_n) = \mathbb{P}(A \cap B). 
\eQe

From this point on, we assume that $\mathbb{P}(B) > 0$ and $\mathbb{P}(B_n) > 0$ for all $n$,
so that the conditional probabilities are well-defined. \\

\smallskip

\textbf{(i)} As $A_n \to A$ and $B \to B$, by the established result, we obtain
\eQb
\lim_{n \to \infty} \mathbb{P}(A_n \cap B) = \mathbb{P}(A \cap B) \> \text{ and } 
\lim_{n \to \infty}\mathbb{P}(B) = \mathbb{P}(B).
\eQe
Hence, by the limit rule, we have that the limit of the given term exists and 
\eQb
\lim_{n \to \infty} \dfrac{\mathbb{P}(A_n \cap B)}{\mathbb{P}(B)} = 
\dfrac{\mathbb{P}(A \cap B)}{\mathbb{P}(B)} = \mathbb{P}(A|B).
\eQe

\pagebreak
 
\textbf{(ii)} As $A \to A$ and $B_n \to B$, by the established result, we obtain
\eQb
\lim_{n \to \infty}\mathbb{P}(A \cap B_n) = \mathbb{P}(A \cap B) \> \text{ and }
\lim_{n \to \infty} \mathbb{P}(B_n) = \mathbb{P}(B). 
\eQe
Hence, by the limit rule, we have that the limit of the given term exists and 
\eQb
\lim_{n \to \infty} \dfrac{\mathbb{P}(A \cap B_n)}{\mathbb{P}(B_n)} = 
\dfrac{\mathbb{P}(A \cap B)}{\mathbb{P}(B)} = \mathbb{P}(A|B).
\eQe

\smallskip

\textbf{(iii)} As $A_n \to A$ and $B_n \to B$, by the established result, we obtain
\eQb
\lim_{n \to \infty} \mathbb{P}(A_n \cap B_n) = \mathbb{P}(A \cap B) \> \text{ and } 
\lim_{n \to \infty}\mathbb{P}(B_n) = \mathbb{P}(B).
\eQe
Hence, by the limit rule, we have that the limit of the given term exists and 
\eQb
\lim_{n \to \infty} \dfrac{\mathbb{P}(A_n \cap B_n)}{\mathbb{P}(B_n)} &=& 
\dfrac{\mathbb{P}(A \cap B)}{\mathbb{P}(B)} = \mathbb{P}(A|B),
\eQe
which completes the proof. $\qed$
\end{solution}

\bigskip

\begin{question}[2. Conditional Probability]
\end{question}
\begin{solution}
As $\mathbb{P}(B \cap C) \neq 0$, by the definition of conditional probability, we have
\eQb
\mathbb{P}(A | B \cap C) &=& \dfrac{\mathbb{P}(A \cap B \cap C)}{\mathbb{P}(B \cap C)}.
\eQe
As $\{ A, B, C\}$ are mutually independent events, we obtain
\eQb
\mathbb{P}(A | B \cap C) &=& \dfrac{\mathbb{P}(A) \mathbb{P}(B) \mathbb{P}(C)}{\mathbb{P}(B)
\mathbb{P}(C)}\\
&=& \mathbb{P}(A),
\eQe
as desired. $\qed$

\end{solution}


\bigskip

\begin{question}[3. Conditional Probability]
\end{question}
\begin{solution}
Given the assumption, we can compute the probability of $\mathbb{P}(A)$ by 
using the conditional probability as  
\eQb
\mathbb{P}(A) &=& \mathbb{P}(A|G=M)\mathbb{P}(G=M) + \mathbb{P}(A|G=F)\mathbb{P}(G=F).
\eQe
Substituting the givens, we have
\eQb
\mathbb{P}(A) &=& 0.3\cdot 0.5 0.2\cdot 0.5 \\ 
&=& 0.25.
\eQe
Similarly, we can express $\mathbb{P}(B)$ as 
\eQb
\mathbb{P}(B) &=& \mathbb{P}(B|G=M)\mathbb{P}(G=M) + \mathbb{P}(B|G=F)\mathbb{P}(G=F).
\eQe
As the claims are made independently each year, by substituting the givens, we have
\eQb
\mathbb{P}(B) &=& 0.3^2 \cdot 0.5 + 0.2^2\cdot 0.5 \\
&=& 0.065.
\eQe
Therefore, we have that
\eQb
\mathbb{P}(B|A) &=& \dfrac{\mathbb{P}(B \cap A)}{\mathbb{P}(A)} \\
&=& \dfrac{\mathbb{P}(B)}{\mathbb{P}(C)} \\
&=& \dfrac{0.0065}{0.25} \\
&=& 0.26. \\
\eQe
The independence assumption was used to compute the probability of the event $B$. It does not
give an explicit information about the conditional probability $\mathbb{P}(B|A)$ without the 
computation. 
As the head of the insurance company, one should minimize the probability of filling a complaint.
Hence, one would prefer one who did not have a claim in the previous year, as the computation shows.
$\qed$
\end{solution}
\bigskip

\begin{question}[4. Genetics]
\end{question}
\begin{solution}
We are given that a parent passes on one of its alleles, chosen at 
random uniformly to its child, and the genotype of the child combines
alleles from both parents. Furthermore, the probability applies
homogenously to the entire population. We then can write a probability
of a parent passing on $A$ allele as 
\eQb
\mathbb{P}(A) &=& P.
\eQe
Then, by the definition of probability, we have
\eQb
\mathbb{P}(a) &=& 1 - P = Q,
\eQe
where $P + Q = 1$, as there are only two options. Hence, the probabilities 
that two parents passing on particular alleles are 
\eQb
\mathbb{P}(AA) = \mathbb{P}(A)\mathbb{P}(A) = P^2 \\
\mathbb{P}(Aa) = 2\mathbb{P}(A)\mathbb{P}(a) = 2PQ \\
\mathbb{P}(aa) = \mathbb{P}(a)\mathbb{P}(a) = Q^2,
\eQe 
as desired. Recall that $P + Q = 1$. Hence, the $p$, $q$, $r$ terms,
which are the probabilities computed above, only depend on one parameter
$P$. $\qed$
\end{solution}

\bigskip

\begin{question}[5. The Moments of Poisson Distribution]
\end{question}
\begin{solution} Let $X$ be a random variable with Poisson distribution with
parameter $\lambda > 0$. We proceed to compute the expectation $\mathbb{E}[X]$.
By the definition of expectation, we have
\eQb
\mathbb{E}[X] &=& \sum_{k=0}^{\infty}
k\dfrac{\lambda^k e^{-\lambda}}{k!} \\
&=& \sum_{k=1}^{\infty} k\dfrac{\lambda^k e^{-\lambda}}{k!}, \\
\eQe
as the $0$-th term vanishes. Pulling out the $\lambda e^{-\lambda}$, we obtain
\eQb
\mathbb{E}[X] 
&=& \lambda e^{-\lambda} 
\sum_{k=1}^{\infty} \dfrac{\lambda^{k-1}}{(k-1)!} \\
&=& \lambda e^{-\lambda} 
\sum_{p=0}^{\infty} \dfrac{\lambda^{p}}{p!}, \\
\eQe
where $p = k-1$. As the series converges to $e^{\lambda}$, we have
\eQb
\mathbb{E}[X] &=& \lambda.
\eQe
We now compute the variance $\mathrm{Var}[X]$. By the definition of variance, we have
\eQb
\mathrm{Var}[X] &=& \mathbb{E}[X^2] - \mathbb{E}[X]^2.
\eQe
Again by the definition of expectation, we have
\eQb
\mathbb{E}[X^2] &=& \sum_{k=0}^{\infty} k^2 \dfrac{\lambda^k e^{-\lambda}}{k!} \\
&=& \sum_{k=1}^{\infty} k^2 \dfrac{\lambda^k e^{-\lambda}}{k!}, 
\eQe
as the $0$ term vanishes. With a series of simple algebraic manipulation, including
the taylor series expansion for exponential as above, we have
\eQb
\mathbb{E}[X^2] &=& \sum_{k=1}^{\infty} k^2 \dfrac{\lambda^k e^{-\lambda}}{k!} \\
&=& \lambda e^{-\lambda}\sum_{k=1}^{\infty} k \dfrac{\lambda^{k-1}}{(k-1)!} \\
&=& \lambda e^{-\lambda}(\sum_{k=1}^{\infty} \dfrac{\lambda^{k-1}}{(k-1)!}
+ \sum_{k=1}^{\infty} (k-1)\dfrac{\lambda^{k-1}}{(k-1)!}) \\
&=& \lambda e^{-\lambda}(\sum_{k=1}^{\infty} \dfrac{\lambda^{k-1}}{(k-1)!}
+ \lambda \sum_{k=2}^{\infty} \dfrac{\lambda^{k-2}}{(k-2)!}) \\
&=& \lambda e^{-\lambda} (e^{\lambda} + \lambda e^{\lambda}) \\
&=& \lambda^2 + \lambda.
\eQe
Consequently, it follows that
\eQb
\mathrm{Var}[X] &=& \lambda.
\eQe

\smallskip

Now, we compute the term $\mathbb{E}[X(X-1)...(X-K+1)]$ for any $k \in \mathbb{N}$.
Note that the above term can be written as $\mathbb{E}\left[ \dfrac{X!}{(X-K)!} \right]$. Now, we can
write the expectation of the $\dfrac{X!}{(X-K)!}$ random variable as
\eQb
\mathbb{E}\left[ \dfrac{X!}{(X-K)!} \right] &=& \sum_{p=K}^{\infty}
\dfrac{p!}{(p-K)!} \dfrac{\lambda^{p}e^{-\lambda}}{p!} \\
&=& e^{-\lambda}\lambda^{K} \sum_{p = K}^{\infty} \dfrac{\lambda^{(p-K)}}{(p-K)!} \\
&=& \lambda^K.
\eQe

\smallskip
The $k$th moment of the distribution can be written as
\eQb
\mathbb{E}[X^k] &=& \sum_{i=1}^{K} \lambda^i S(K,i),
\eQe
where $S(K,i)$ denotes the Striling number of second kind, which computes the number of unlabelled
partitions out of labeled objects. We esssentially get the number of partions for the falling factorial
of size $i$. $\qed$
\end{solution}
\bigskip

\begin{question}[6. Memoryless Property of the Geometric Distribution]
\end{question}
\begin{solution}
We first compute an explicit formula for the $\mathbb{P}(X > k)$ term with an arbitrary $k \geq 0$.
By the definition of the geometric random variable and geometric series formula, we have
\eQb
\mathbb{P}(X > k) &=& 1 - \sum_{i=0}^{k} \mathbb{P}(X = i) \\
&=& 1 - \sum_{i=0}^{k} (1-q)^{i}q \\
&=& 1 - q \sum_{i=0}^{k} (1-q)^{i} \\
&=& 1 - q\dfrac{1 - (1-q)^{k+1}}{q} \\
&=& (1-q)^{k+1}
\eQe
By the above formula, we can write the LHS as
\eQb
\mathbb{P}(X > i + j | X \geq i) &=& \dfrac{\mathbb{P}(X > i + j , X > i)}{\mathbb{P}(X \geq i)} \\
&=& \dfrac{\mathbb{P}(X > i + j)}{\mathbb{P}(X > i - 1)} \\
&=& \dfrac{(1-q)^{i+j+1}}{(1-q)^{i}} \\
&=& (1-q)^{j+1},
\eQe
for $i,j > 0$.
Again, by the above formula, we can write the RHS as 
\eQb
\mathbb{P}(X > j) &=& (1-q)^{j+1},
\eQe
for $i, j > 0$.
Hence, we have shown that the given memoryless property hold for $i,j > 0$. $\qed$

\end{solution}


\end{document}





