\documentclass{article} % For LaTeX2e
\usepackage{nips14submit_e,times}
\usepackage{amsmath}
\usepackage{amsthm}
\usepackage{amssymb}
\usepackage{mathtools}
\usepackage{hyperref}
\usepackage{url}
\usepackage{mathrsfs}
\usepackage{algorithm}
\usepackage[noend]{algpseudocode}
%\documentstyle[nips14submit_09,times,art10]{article} % For LaTeX 2.09

\usepackage{graphicx}
\usepackage{caption}
\usepackage{subcaption}

\def\eQb#1\eQe{\begin{eqnarray*}#1\end{eqnarray*}}
\def\eQnb#1\eQne{\begin{eqnarray}#1\end{eqnarray}}
\providecommand{\e}[1]{\ensuremath{\times 10^{#1}}}
\providecommand{\pb}[0]{\pagebreak}

\usepackage{mathtools}
\DeclarePairedDelimiter{\ceil}{\lceil}{\rceil}
\DeclarePairedDelimiter{\floor}{\lfloor}{\rfloor}

\newcommand{\E}{\mathrm{E}}
\newcommand{\Var}{\mathrm{Var}}
\newcommand{\Cov}{\mathrm{Cov}}

\def\Qb#1\Qe{\begin{question}#1\end{question}}
\def\Sb#1\Se{\begin{solution}#1\end{solution}}

\newenvironment{claim}[1]{\par\noindent\underline{Claim:}\space#1}{}
\newtheoremstyle{quest}{\topsep}{\topsep}{}{}{\bfseries}{}{ }{\thmname{#1}\thmnote{ #3}.}
\theoremstyle{quest}
\newtheorem*{definition}{Definition}
\newtheorem*{theorem}{Theorem}
\newtheorem*{lemma}{Lemma}
\newtheorem*{question}{Question}
\newtheorem*{preposition}{Preposition}
\newtheorem*{exercise}{Exercise}
\newtheorem*{challengeproblem}{Challenge Problem}
\newtheorem*{solution}{Solution}
\newtheorem*{remark}{Remark}
\usepackage{verbatimbox}
\usepackage{listings}
\title{Basic Probability: \\
Problem Set V}


\author{
Youngduck Choi \\
CILVR Lab \\
New York University\\
\texttt{yc1104@nyu.edu} \\
}


% The \author macro works with any number of authors. There are two commands
% used to separate the names and addresses of multiple authors: \And and \AND.
%
% Using \And between authors leaves it to \LaTeX{} to determine where to break
% the lines. Using \AND forces a linebreak at that point. So, if \LaTeX{}
% puts 3 of 4 authors names on the first line, and the last on the second
% line, try using \AND instead of \And before the third author name.

\newcommand{\fix}{\marginpar{FIX}}
\newcommand{\new}{\marginpar{NEW}}

\nipsfinalcopy % Uncomment for camera-ready version

\begin{document}


\maketitle

\begin{abstract}
This work contains a collection of solutions for selected problems 
of the Basic Probability course of Fall 2015.
\end{abstract}

\bigskip

\begin{question}[1]
\end{question}
\begin{solution}
\textbf{(i)} By the definition of Dirac Distribution and the linearity, we have
\eQb
E[X] &=& \int_{-\infty}^{\infty} xf_{X}(x) dx \\
&=& \int_{-\infty}^{\infty} x(p\delta_a(x)) + q\delta_b(x))dx \\ 
&=& p\int_{-\infty}^{\infty} x\delta_a(x)) dx + q\int_{-\infty}^{\infty} x\delta_b(x) dx \\
&=& pa + qb,
\eQe
as expected.

\smallskip

\textbf{(ii)} By definition of Poisson Distribution, we have
\eQb
E[X] &=& \sum_{k=0}^{\infty} k\dfrac{1}{k!}\lambda^k e^{-\lambda} \\
&=& \lambda e^{-\lambda} \sum_{k=0}^{\infty} k\dfrac{1}{k!} \lambda^{k-1} \\
&=& \lambda,
\eQe
as expected. $\qed$
\end{solution}

\bigskip

\begin{question}[2]
\end{question}
\begin{solution}
Let $X$ be uniformly distributed on $[0,1]$. Then, the distribution
of $X$ can be written as 
\eQb
F_X(x) = 
 \begin{cases} 0 \>\>\> \mbox{for } x \leq 0  \\ 
x \>\>\> \mbox{for } x \in (0,1) \\
1 \>\>\> \mbox{for } x \geq 1. \\
\end{cases} 
\eQe
Now, consider the random variable $-\lambda \log X$, for $\lambda > 0$. 
Observe that for $ x > 0$,
\eQb
-\lambda\log(y) \leq x \iff y \geq  e^{-\frac{x}{\lambda}}.
\eQe
Consequently, the distribution of $-\lambda \log X$ can be written as
\eQb
F_{-\lambda \log X}(x) = 
 \begin{cases} 
0 \>\>\> \mbox{for } x \in (-\infty,0) \\
1 - e^{-\lambda x} \>\>\> \mbox{for } x \in [0,\infty) \\
\end{cases} ,
\eQe
which is precisely the exponential distribution as desired. $\qed$
\end{solution}

\bigskip

\begin{question}[3]
\end{question}
\begin{solution}
Let $X$ be a standard Gaussian random variable. The density then
can be written as
\eQb
f_X(x) = \dfrac{1}{2\pi}e^{-x^2}.
\eQe
Notice that for the $\dfrac{1}{X^2}$ random variable, the density
can be written as
\eQb
f_{\frac{1}{X^2}}(x) &=& f_{X}(\dfrac{1}{\sqrt{x}}) 
+ f_{X}(-\dfrac{1}{\sqrt{x}}) \\
&=& \dfrac{1}{\pi}e^{\frac{1}{x}}, 
\eQe 
as desired. $\qed$

\end{solution}

\bigskip

\begin{question}[4]
\end{question}
\begin{solution}
\textbf{(i)} We have $\Omega = \{H,T \}^N$, $\mathscr{A} = 2^{\Omega}$,
and $\mathbb{P}:\mathscr{A} \to [0,1]$ such that 
$\mathbb{P}(A) = \dfrac{|A|}{2^N}$, for any $A \in \mathscr{A}$.

\smallskip

\textbf{(ii)} Define $A_n$ be a sequence of events such that
the pattern $(H,H,T,H,T,H,H)$ occurs with the $n$ toss experiment,
and $p(A_n)$ be the probability of $A_n$. Observe that we can identify
an event $L_n$ where 
$(H,H,T,H,T,H,H)$ occurring from the $1$st index to the $7$th
index with the $n$ toss experiment precisely. The probability of
such event is $\dfrac{2^{n-7}}{2^n}$ for $n \geq 7$. Observe that
$\lim_{n \to \infty} p(L_n) = 1$. Since $L_n$ is a subset of $A_n$,
by the monotonicity property of probability measure, we have
$p(L_n) \leq p(A_n) \leq 1$ for all $n$. Therefore, by the Squeeze 
Theorem, we have that $\lim_{n \to \infty} p(A_n) = 1$. $\qed$ 
 

\end{solution}

\bigskip

\begin{question}[5]
\end{question}
\begin{solution}

\end{solution}

\bigskip

\begin{question}[6]
\end{question}
\begin{solution}
As $c > 0$ and $\delta > 0$, by the Markov inequality, we have
\eQb
P(|X| > \delta) &\leq& \dfrac{E[e^{\lambda |X|}]}{e^{\lambda \delta}}
\eQe
\end{solution}


\end{document}

