\documentclass{article} % For LaTeX2e
\usepackage{nips14submit_e,times}
\usepackage{amsmath}
\usepackage{amsthm}
\usepackage{amssymb}
\usepackage{mathtools}
\usepackage{hyperref}
\usepackage{url}
\usepackage{algorithm}
\usepackage[noend]{algpseudocode}
%\documentstyle[nips14submit_09,times,art10]{article} % For LaTeX 2.09

\usepackage{graphicx}
\usepackage{caption}
\usepackage{subcaption}

\def\eQb#1\eQe{\begin{eqnarray*}#1\end{eqnarray*}}
\def\eQnb#1\eQne{\begin{eqnarray}#1\end{eqnarray}}
\providecommand{\e}[1]{\ensuremath{\times 10^{#1}}}
\providecommand{\pb}[0]{\pagebreak}

\newcommand{\E}{\mathrm{E}}
\newcommand{\Var}{\mathrm{Var}}
\newcommand{\Cov}{\mathrm{Cov}}

\def\Qb#1\Qe{\begin{question}#1\end{question}}
\def\Sb#1\Se{\begin{solution}#1\end{solution}}

\newenvironment{claim}[1]{\par\noindent\underline{Claim:}\space#1}{}
\newtheoremstyle{quest}{\topsep}{\topsep}{}{}{\bfseries}{}{ }{\thmname{#1}\thmnote{ #3}.}
\theoremstyle{quest}
\newtheorem*{definition}{Definition}
\newtheorem*{theorem}{Theorem}
\newtheorem*{lemma}{Lemma}
\newtheorem*{question}{Question}
\newtheorem*{preposition}{Preposition}
\newtheorem*{exercise}{Exercise}
\newtheorem*{challengeproblem}{Challenge Problem}
\newtheorem*{solution}{Solution}
\newtheorem*{remark}{Remark}
\usepackage{verbatimbox}
\usepackage{listings}
\title{Basic Probability: \\
Problem Set I}


\author{
Youngduck Choi \\
CILVR Lab \\
New York University\\
\texttt{yc1104@nyu.edu} \\
}


% The \author macro works with any number of authors. There are two commands
% used to separate the names and addresses of multiple authors: \And and \AND.
%
% Using \And between authors leaves it to \LaTeX{} to determine where to break
% the lines. Using \AND forces a linebreak at that point. So, if \LaTeX{}
% puts 3 of 4 authors names on the first line, and the last on the second
% line, try using \AND instead of \And before the third author name.

\newcommand{\fix}{\marginpar{FIX}}
\newcommand{\new}{\marginpar{NEW}}

\nipsfinalcopy % Uncomment for camera-ready version

\begin{document}


\maketitle

\begin{abstract}
This work contains a collection of solutions for selected problems 
of the Basic Probability course of Fall 2015.
\end{abstract}

\begin{question}[1. Countability]
\end{question}
\begin{solution}
\textbf{(i)}
Let $X = \{ (a,b) | a \in \mathbb{Q}, b \in \mathbb{Q} \}$. In particular, $X = \mathbb{Q}
\times \mathbb{Q}$. Notice that by the Cantor diagonalization argument, there exists
a bijective map $\phi: \mathbb{Q} \times \mathbb{Q} \to \mathbb{N} \times \mathbb{N}$. Now,
consider a map $\psi: \mathbb{N} \times \mathbb{N} \to \mathbb{N}$ such that $\psi(m,n) = (m+n)^2 + n$.
With some algebra, we can show that if $\psi(m,n) = \psi(m',n')$, then $m = m'$ and $n = n'$. Hence,
$\psi$ is injective, and $\mathbb{N} \times \mathbb{N}$ is equipotent to $\psi(\mathbb{N} \times \mathbb{N})$,
which is a subset of the countable set $\mathbb{N}$. Therefore, we have shown that $\mathbb{N} \times \mathbb{N}$
is countable, and consequently $X$ is countable, due to the existence of the bijective map $\phi$.

\bigskip

\textbf{(ii)}
Let $X = \{ B((x,y),r) | x = y \}$. Consider the set $Y$ such that $Y = \{ B((x,y),1) | x = y \}$. $Y$ is in fact
equipotent with $\mathbb{R}$, as you can arbitrarily pick the center from $\mathbb{R}$ and that determines the
circle in $Y$. Hence, $Y$ is uncountable. Furthermore, notice that $Y \subset X$. Hence, $X$ is uncountable.

\bigskip

\textbf{(iii)}
We can prove that the set of all sequences of integers whose term are either $0$ or $1$ is uncountable,
by the diagnonalization argument. Assume that the set is countable. Then, let $\{ s_i \}_{i=1}^{\infty}$
be an enumeration of such set, such that each $s_i$ is a sequence, further denoted by
$\{ s_{i,k} \}_{k=1}^{\infty}$. Construct a sequence $\{ x_i \}_{i=1}^{\infty}$ such that $x_i = s_{i,i}$.
Then, $\{ x_i \}$ sequence does not belong to the set of all sequences integers whose term are 
either $0$ or $1$. Contradiction. The set is uncountable.

\end{solution}

\bigskip

\begin{question}[2. Basic Combinatorics I]
\end{question}
\begin{solution}

\textbf{(i)} The total number of possible ordering of age is $5!$, as we treat each children to be distinct.
The total number of possible ordering of age, given that the eldest three children all have to girls
is $3!2!$. Hence, the probability of the event, where the eldest three children are girls, is 
$\dfrac{3!2!}{5!} = \dfrac{1}{\binom{5}{2}}$, given the assumption that all orderings are equally likely.

\bigskip

\textbf{(ii)}
The total number of possible team arrangements, given the constraints is $\binom{11}{5} \binom{6}{4}$. A 
combinatorial interpretation is that we first pick the team with 5 members, then pick the team with 4 members
from the remaining people. As $5$, $4$, and $2$ are all distinct numbers, we do not have to consider
over-counting in this case.

\end{solution}

\pagebreak

\begin{question}[3. Basic Combinatorics II]
\end{question}

\begin{solution}
As there are $30$ distinct classes, we have $\binom{30}{7}$ for the numerator. For the denominator,
the total number of possible configuration, with which we have a class everyday,
we first have $6^5$ to choose a class for Monday through Friday. Now, for the remaining two classes,
we can freely choose from the free $2$ slots. Hence, we have $\binom{25}{2}$, and $6^5 \binom{25}{2}$
for the total number. Therefore, the probability, under uniform randomness, is $\dfrac{6^5
\binom{25}{2}}{\binom{30}{7}}$. 
\end{solution}

\bigskip

\begin{question}[4. Geometric Distribution and its Variance]
\end{question}
\begin{solution}
We have that $\mathrm{Var}(X) = \mathrm{E}[X^2] - \mathrm{E}[X]^2$. We proceed to compute 
$\mathrm{E}[X]$,
for the given geometric distribution, $\mathbb{P}(X = k) = (1-q)^k q$. 
\eQb
\mathrm{E}[X] &=& \sum_{k=0}^{\infty} k(1-q)^{k}q \\
&=& q(1-q)\sum_{k=1}^{\infty} k(p)^{k-1},
\eQe
where $p = 1-q$. Notice that the summation is a derivative of the power series $\sum p^k$. Hence,
we can simplify the RHS and obtain
\eQb
\mathrm{E}[X] &=& q(1-q)\dfrac{1}{(1-p)^2} \nonumber \\
&=& \dfrac{1-q}{q}.
\eQe
We now proceed to compute $\mathrm{E}[X^2]$. We first have
\eQb
\mathrm{E}[X^2] &=& \sum_{k=0}^{\infty} k^2 (1-q)^k q \\
&=& q(1-q)^2\sum_{k=2}^{\infty} k(k-1)(p)^{k-1} - 
q(1-q) \sum_{k=1}^{\infty}
k(p)^{k-1},
\eQe
where $p = 1-q$. Again using the derivative of the power series $\sum p_k$ we have,
\eQb
\mathrm{E}[X^2] &=& q(1-q)^2(\dfrac{2}{(1-p)^3}) - q(1-q)(\dfrac{1}{(1-p)^2}) \\
&=& \dfrac{2(1-q)^2}{q^2} - \dfrac{1-q}{q} \\ 
&=& \dfrac{(2-q)(1-q)}{q^2}.
\eQe

Substituting the computed 
equality into the variance equation, we obtain
\eQb
\mathrm{Var}(X) &=& \dfrac{(2-q)(1-q)}{q^2} - (\dfrac{1-q}{q})^2 
\eQe
Hence, we have that
\eQb
\mathrm{Var}(X) &=& \dfrac{1-q}{q^2}.
\eQe

\end{solution}

\bigskip

\begin{question}[5. $\sigma$-algebra generated]
\end{question}
\begin{solution}
Assume $\emptyset \subsetneq A \subsetneq B \subsetneq \Omega$.
The $\sigma$-algebra generated by $\{ A, B \}$ can be written as
\[
\sigma( \{ A, B \} ) = \{ \emptyset , \> A , \> B , \> A^C, \> B^C, \> A\cup B^C, \> A^C \cap B, \> \Omega \}.
\]
\end{solution}

\pagebreak

\begin{question}[6. Stirling's Formula]
\end{question}
\begin{solution}
We want to compute $\underset{n \to \infty}{\lim} \sum_{k=1}^{n} ln(k)$.
This can be re-written as $\underset{n \to \infty}{\lim} ln(k!)$. Using the trapezoid approximation,
we have that 
\eQb
ln(n!) \approx \int_{1}^{n} ln(x)dx = nln(n) - n + 1.
\eQe
\end{solution}

\end{document}
