\documentclass{article} % For LaTeX2e
\usepackage{nips14submit_e,times}
\usepackage{amsmath}
\usepackage{amsthm}
\usepackage{amssymb}
\usepackage{mathtools}
\usepackage{hyperref}
\usepackage{url}
\usepackage{mathrsfs}
\usepackage{algorithm}
\usepackage[noend]{algpseudocode}
%\documentstyle[nips14submit_09,times,art10]{article} % For LaTeX 2.09

\usepackage{graphicx}
\usepackage{caption}
\usepackage{subcaption}

\def\eQb#1\eQe{\begin{eqnarray*}#1\end{eqnarray*}}
\def\eQnb#1\eQne{\begin{eqnarray}#1\end{eqnarray}}
\providecommand{\e}[1]{\ensuremath{\times 10^{#1}}}
\providecommand{\pb}[0]{\pagebreak}

\usepackage{mathtools}
\DeclarePairedDelimiter{\ceil}{\lceil}{\rceil}
\DeclarePairedDelimiter{\floor}{\lfloor}{\rfloor}

\newcommand{\E}{\mathrm{E}}
\newcommand{\Var}{\mathrm{Var}}
\newcommand{\Cov}{\mathrm{Cov}}

\def\Qb#1\Qe{\begin{question}#1\end{question}}
\def\Sb#1\Se{\begin{solution}#1\end{solution}}

\newenvironment{claim}[1]{\par\noindent\underline{Claim:}\space#1}{}
\newtheoremstyle{quest}{\topsep}{\topsep}{}{}{\bfseries}{}{ }{\thmname{#1}\thmnote{ #3}.}
\theoremstyle{quest}
\newtheorem*{definition}{Definition}
\newtheorem*{theorem}{Theorem}
\newtheorem*{lemma}{Lemma}
\newtheorem*{question}{Question}
\newtheorem*{preposition}{Preposition}
\newtheorem*{exercise}{Exercise}
\newtheorem*{challengeproblem}{Challenge Problem}
\newtheorem*{solution}{Solution}
\newtheorem*{remark}{Remark}
\usepackage{verbatimbox}
\usepackage{listings}
\title{Basic Probability: \\
Problem Set II}


\author{
Youngduck Choi \\
CILVR Lab \\
New York University\\
\texttt{yc1104@nyu.edu} \\
}


% The \author macro works with any number of authors. There are two commands
% used to separate the names and addresses of multiple authors: \And and \AND.
%
% Using \And between authors leaves it to \LaTeX{} to determine where to break
% the lines. Using \AND forces a linebreak at that point. So, if \LaTeX{}
% puts 3 of 4 authors names on the first line, and the last on the second
% line, try using \AND instead of \And before the third author name.

\newcommand{\fix}{\marginpar{FIX}}
\newcommand{\new}{\marginpar{NEW}}

\nipsfinalcopy % Uncomment for camera-ready version

\begin{document}


\maketitle

\begin{abstract}
This work contains a collection of solutions for selected problems 
of the Basic Probability course of Fall 2015.
\end{abstract}

\begin{question}[1. Countability and $\sigma$-algebra]
\end{question}
\begin{solution}
Suppose $\Omega$ is a infinite set. Let $\mathscr{A}$ be a sub-collection of $\mathbb{P}(\Omega)$, 
defined by 
\eQb
\mathscr{A} &=& \{ X \subseteq \Omega \> | \> X \text{ is finite or } X^c \text{ is finite } \}. 
\eQe
As $\Omega$ is infinite, there exists an injective map $\phi:\mathbb{N} \to \Omega$. Consider
the two images of $\phi$: $\phi(2\mathbb{N})$ and $\phi(2\mathbb{N}+1)$, where $2\mathbb{N}$
denotes the set of evens and $2\mathbb{N}+1$ denotes the set of odds. Observe that
$\phi(2\mathbb{N})$ is infinite and $\phi(2\mathbb{N})^c$ is infinite, as $\phi$ is injective.
Hence, $\phi(2\mathbb{N}) \notin \mathscr{A}$. The image, however, can be expressed in the following
way:
\eQb
\phi(2\mathbb{N}) &=& \bigcup_{k \in 2\mathbb{N}} \{ \phi(k) \},
\eQe
which gives that $\phi(2\mathbb{N})$ is a countable union of sets in $\mathscr{A}$, as
each point set $\{ \phi(k) \} $ is finite and in $\mathscr{A}$.
Therefore, by the definition of $\sigma$-algebra, we obtain $\phi(2\mathbb{N}) \in \mathscr{A}$.
This is a contradiction.
Hence, $\mathscr{A}$, defined by the
given characterization, is not a $\sigma$-algebra, when
$\Omega$ is infinite. $\qed$

\end{solution}

\bigskip

\begin{question}[2. Limit of Probabilities of Disjoint Events]
\end{question}
\begin{solution}
Let $( A_n )_{n \geq 0}$ be a set of disjoint events and $\mathbb{P}$ be a probability.
Suppose for sake of contradiction that $\underset{n \to \infty}{\lim} \mathbb{P}(A_n )$
does not converge to $0$. Then, there exists $\epsilon > 0$ such that for all $n \geq 0$,
such that $\mathbb{P}(A_n ) \geq \epsilon$. Then, 
as the events are disjoint, by the countable additivity of probability,
we have 
\eQb
\mathbb{P}(\cup_{n=0}^{\infty} A_n ) &=& \sum_{n=0}^{\infty} \mathbb{P}(A_n ) 
\geq \sum_{n=0}^{\infty} \epsilon = \infty. 
\eQe
Hence, we obtain that $\mathbb{P}(\cup_{n=0}^{\infty} A_n) \geq \infty$. This is a contradiction,
as a probability measure assigns any event in the $\sigma$-algebra to some real number in $[0,1]$.
Hence, $\underset{n \to \infty}{\lim} \mathbb{P}(A_n) = 0$. $\qed$
\end{solution}

\pagebreak

\begin{question}[3.Bonferroni Inequalities]
\end{question}
\begin{solution}
Let $\{A_i \}$ be a sequence of events. 
We wish to prove the following probalistic inequality:
\eQb
\mathbb{P}(\bigcup_{i=1}^{n} A_i ) &\geq& \sum_{i=1}^{n} \mathbb{P}(A_i) - \sum_{i < j} 
\mathbb{P}(A_i \cap A_j).
\eQe
We proceed by induction. Let $A_1$ and $A_2$ be two events. Then, by the finite additivity of 
probability measure, we have
\eQb
\mathbb{P}(A_1 \cup A_2) = \mathbb{P}(A_1) + \mathbb{P}(A_2) - \mathbb{P}(A_1 \cap A_2). 
\eQe
Hence, the inequality holds for $n=2$ case. Now, assume that the inequality holds for some $k$.
\end{solution}

\bigskip

\begin{question}[4.]
\end{question}
\begin{solution}
A pair of dice is rolled until a sum of either 
$5$ or $7$ appears. We wish to compute the probability that a $5$
occurs first.
Let $E_n$ denote the event, where a $5$ occurs on the $n$th roll,
and no $5$ or $7$ occurs on the first $(n-1)$ roll.
The probability that we wish to compute can be written as
\eQb
\sum_{n=1}^{\infty} \mathbb{P}( E_n ),
\eQe
as each $E_n$ are pairwise disjoint events.
Let $F_n$ denote the event, where no $5$ or $7$ occurs on the $n$th roll,
and let $G_n$ denote the event, where $5$ occurs on the $n$th roll.
\eQb
\mathbb{P}(E_n ) &=& \mathbb{P}((\cup_{k=1}^{n-1}F_k) \cup G_n). 
\eQe
Now, with the independence assumption on each throw, we can factorize the RHS,
and obtain
\eQnb
\mathbb{P}(E_n ) &=& \mathbb{P}(G_n) \prod_{k=1}^{n-1}\mathbb{P}(F_{k}), 
\eQne
where $n-1 = 0$ case for the product term is defined to be $1$.
Assuming that the two dices are both fair dices, through a simple combinatorial argument(just count!),
we can obtain that
\eQb
\mathbb{P}(G_n) &=& \dfrac{4}{36} = \dfrac{1}{9}, \\
\mathbb{P}(F_n) &=& 1 - \dfrac{4}{36} - \dfrac{6}{36} = \dfrac{26}{36} = \dfrac{13}{18}. 
\eQe
Substituting the above equations into $(1)$, we have
\eQb
\mathbb{P}(E_n) &=& \dfrac{1}{9}\prod_{k=1}^{n-1} \dfrac{13}{18} \\
&=& \dfrac{1}{9} (\dfrac{13}{18})^{n-1}
\eQe
Consequently, we obtain
\eQb
\sum_{n=1}^{\infty}\mathbb{P}(E_n) &=& \sum_{n=1}^{\infty}
\dfrac{1}{9} (\dfrac{13}{18})^{n-1} \\
&=& \dfrac{1}{9} (\dfrac{1}{1 - \frac{13}{18}}) \\
&=& \dfrac{2}{5}.
\eQe
Therefore, the probability that a $5$ occurs first is $\dfrac{2}{5}$. $\qed$

\end{solution}

\bigskip

\begin{question}[5. $\limsup$ and $\liminf$ ]
\end{question}
\begin{solution}
Let $\mathbb{P}$ be a probability measure on $\Omega$ endowed with a 
$\sigma$-algebra $\mathscr{A}$. We then define $\limsup$ and $\liminf$
of a sequence of events $\{ A_n \}$, chosen from $\mathscr{A}$ as follow:
\eQb
\underset{n \to \infty}{\limsup} \{ A_n \} &=& \cap_{n=1}^{\infty} \cup_{m \geq n} A_n, \\
\underset{n \to \infty}{\liminf} \{ A_n \} &=& \cup_{n=1}^{\infty} \cap_{m \geq n} A_n. \\
\eQe

\smallskip

\textbf{(i)} The meaning of the above events can be written as
\eQb
\underset{n \to \infty}{\limsup} \{ A_n \} &=& \bigcap_{n=1}^{\infty} \bigcup_{m \geq n} A_n \\
&=& \{ x \in \Omega \> | \> \forall n , \exists m \geq n \text{ such that } x \in A_m \} \\ 
&=& \{ x \in \Omega \> | \> \text{the outcome } x 
\text{ appears infinitely often in the event sequence}\}, \\ 
\underset{n \to \infty}{\liminf} \{ A_n \} &=& \bigcup_{n=1}^{\infty} \bigcap_{m \geq n} A_n \\
&=& \{ x \in \Omega \> | \> \exists n , \forall m \geq n , x \in A_m \}. \\ 
&=& \{ x \in \Omega \> | 
\> \text{the outcome } x 
\text{ does not appear only for a} \\
&& \text{finite number of events in the sequence} \}. 
\eQe

\smallskip

\textbf{(ii)} 
Let $\Omega = \mathbb{R}$, and $\mathscr{A}$ is the Borel $\sigma-$algebra. For any $ p \leq 1$,
we define
\eQb 
A_{2p} = {\Big [} -1, 2 + \dfrac{1}{2p}{\Big )}, \> \> A_{2p+1} = {\Big (}-2- 
\dfrac{1}{2p+1}, 1 {\Big ]}. 
\eQe
We first denote the $\limsup$ set as $S$ and $\liminf$ set as $I$. Then, we claim that
\eQb
S &=& \underset{n \to \infty}{\limsup} \{ A_n \} = {\Big [} -2, 2 {\Big ]}, \\
I &=& \underset{n \to \infty}{\liminf} \{ A_n \} = {\Big [} -1, 1 {\Big ]},
\eQe
which we denote as $S$ and $I$ respectively.

\smallskip

We first prove that $S = [-2,2]$. Consider $x \in [0,2]$. Observe that for any $m$, $x \in A_{2\ceil{
\frac{m}{2}}}$, as $[0,2] \subseteq [-1,2+\dfrac{1}{2\ceil{\frac{m}{2}}})$. As $2\ceil{
\frac{m}{2}} \geq m$, we have $[0,2] \subseteq  S$. Similarly, consider $x \in [-2,0]$.
Observe that for any $m$, $x \in A_{2\frac{\floor{m}}{2}+1}$, as $[-2,0] \subseteq 
(-2-\dfrac{1}{2\floor{\frac{m}{2}}+1},1]$. As $2\floor{\frac{m}{2}} \geq m$, we have $[-2,0] 
\subseteq S$. Suppose now that $x > 2$. Then, by the Archemedian property of real number, 
there exists an integer $p$ such that $2+\dfrac{1}{2p} < x$. Hence, $x \notin A_k$ for all $k \geq 2p$.
Therefore, for $x > 2$, $x \notin S$. Analogously, for $x < -2$, $x \notin S$. We have shown that
$S = [-2,2]$. 

\smallskip

We now show that $I = [-1,1]$. Observe that $[-1,1] \subseteq A_k$ for all positive integer $k$ 
greater than $1$. Therefore, $[-1,1] \subseteq I$. For $x > 1$, the odd terms do not contain $x$.
For $x < -1$, the even terms do not contain $x$. Hence, we have shown that
$I = [-1,1]$ as desired.

\smallskip

\textbf{(iii)} 

\end{solution}

\bigskip

\begin{question}[6.Zeta Function in Probability]
\end{question}
\begin{solution}
Let $n$ and $m$ be random numbers chosen independently and uniformly on $[[1,N]]$.
Then, we can characterize $\Omega$, and $\mathscr{A}$, which
all implicitly depend on $N$ as follow:
\eQb
\Omega &=& [[1, N]] \times [[1, N]], \\
\mathscr{A} &=& 2^{\Omega},
\eQe
where $2^{\Omega}$ denotes the power set of $\Omega$. Furthermore, 
with the uniform probability assumption, we can define the 
probability measure on the above measurable space $\mathbb{P}: \mathscr{A} \to [0,1]$ by
\eQb
\mathbb{P}(A) = \dfrac{|A|}{N^2} \text{ for } A \in \mathscr{A}.
\eQe
Now, define set of events $A_p$ as follow:
\[
\{ x \in \Omega \> | \> \text{both n and m have a prime } p \text{ as its factor} \}.
\]
Notice that the $(n,m) = 1$ event, for a fixed $N$, can be expressed in terms of $A_p$s as
\eQb
\mathbb{P}((n,m) = 1) = \mathbb{P}(\bigcap_{p \in \bar{P}}A_p^c),
\eQe
where $\bar{P}$ denotes the set of primes not greater than $N$. As $\{ A_p \}$ collection forms
an independent collection, arising from the uniform probability assumption and independence,
the probability can be factorized into
\eQb
\mathbb{P}((n,m) = 1) &=& \prod_{p \in \bar{P}} \mathbb{P}(A_p^c),
\eQe
with $\mathbb{P}(A_p^c) = 1 - \frac{\floor{\frac{N}{p}}}{p}^2$.
Hence, we can re-write the above equation as
\eQb
\mathbb{P}((n,m) = 1) &=& 
\prod_{p \in \bar{P}} 1 -\frac{\floor{\frac{N}{p}}}{p}^2. 
\eQe
Now, as $N \to \infty$, we have 
\eQb
\mathbb{P}((n,m) = 1)_{N \to \infty} &=& 
\lim_{n \to \infty}\prod_{p \in \bar{P}} 1 -\frac{\floor{\frac{N}{p}}}{p}^2 \\ 
&=& \prod_{p \in P} 1 - \dfrac{1}{p^2} \\
&=& \zeta(2)^{-1} \\
&=& \dfrac{6}{\pi^2}, \\
\eQe
where $P$ denotes the set of primes in the second equation, as desired. $\qed$

\end{solution}

\bigskip



\end{document}
