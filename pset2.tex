\documentclass{article} % For LaTeX2e
\usepackage{nips14submit_e,times}
\usepackage{amsmath}
\usepackage{amsthm}
\usepackage{amssymb}
\usepackage{mathtools}
\usepackage{hyperref}
\usepackage{url}
\usepackage{algorithm}
\usepackage[noend]{algpseudocode}
%\documentstyle[nips14submit_09,times,art10]{article} % For LaTeX 2.09

\usepackage{graphicx}
\usepackage{caption}
\usepackage{subcaption}

\def\eQb#1\eQe{\begin{eqnarray*}#1\end{eqnarray*}}
\def\eQnb#1\eQne{\begin{eqnarray}#1\end{eqnarray}}
\providecommand{\e}[1]{\ensuremath{\times 10^{#1}}}
\providecommand{\pb}[0]{\pagebreak}

\newcommand{\E}{\mathrm{E}}
\newcommand{\Var}{\mathrm{Var}}
\newcommand{\Cov}{\mathrm{Cov}}

\def\Qb#1\Qe{\begin{question}#1\end{question}}
\def\Sb#1\Se{\begin{solution}#1\end{solution}}

\newenvironment{claim}[1]{\par\noindent\underline{Claim:}\space#1}{}
\newtheoremstyle{quest}{\topsep}{\topsep}{}{}{\bfseries}{}{ }{\thmname{#1}\thmnote{ #3}.}
\theoremstyle{quest}
\newtheorem*{definition}{Definition}
\newtheorem*{theorem}{Theorem}
\newtheorem*{lemma}{Lemma}
\newtheorem*{question}{Question}
\newtheorem*{preposition}{Preposition}
\newtheorem*{exercise}{Exercise}
\newtheorem*{challengeproblem}{Challenge Problem}
\newtheorem*{solution}{Solution}
\newtheorem*{remark}{Remark}
\usepackage{verbatimbox}
\usepackage{listings}
\title{Basic Probability: \\
Problem Set II}


\author{
Youngduck Choi \\
CILVR Lab \\
New York University\\
\texttt{yc1104@nyu.edu} \\
}


% The \author macro works with any number of authors. There are two commands
% used to separate the names and addresses of multiple authors: \And and \AND.
%
% Using \And between authors leaves it to \LaTeX{} to determine where to break
% the lines. Using \AND forces a linebreak at that point. So, if \LaTeX{}
% puts 3 of 4 authors names on the first line, and the last on the second
% line, try using \AND instead of \And before the third author name.

\newcommand{\fix}{\marginpar{FIX}}
\newcommand{\new}{\marginpar{NEW}}

\nipsfinalcopy % Uncomment for camera-ready version

\begin{document}


\maketitle

\begin{abstract}
This work contains a collection of solutions for selected problems 
of the Basic Probability course of Fall 2015.
\end{abstract}

\begin{question}[1. Countability]
\end{question}
\begin{solution}
Suppose $\Omega$ is a infinite set. Then, there 
\end{solution}

\bigskip

\begin{question}[2. Limit of Probabilities of Disjoint Events]
\end{question}
\begin{solution}
Let $( A_n )_{n \geq 0}$ be a set of disjoint events and $\mathbb{P}$ be a probability.
Suppose for sake of contradiction that $\underset{n \to \infty}{\lim} \mathbb{P}(A_n )$
does not converge to $0$. Then, there exists $\epsilon > 0$ such that for all $n \geq 0$,
such that $\mathbb{P}(A_n ) \geq \epsilon$. Then, 
as the events are disjoint, by the countable additivity of probability,
we have 
\eQb
\mathbb{P}(\cup_{n=0}^{\infty} A_n ) &=& \sum_{n=0}^{\infty} \mathbb{P}(A_n ) 
\geq \sum_{n=0}^{\infty} \epsilon = \infty. 
\eQe
Hence, we obtain that $\mathbb{P}(\cup_{n=0}^{\infty} A_n) \geq \infty$. This is a contradiction,
as a probability measure assigns any event in the $\sigma$-algebra to some real number in $[0,1]$.
Hence, $\underset{n \to \infty}{\lim} \mathbb{P}(A_n) = 0$. $\qed$
\end{solution}

\bigskip

\begin{question}[3.]
\end{question}
\begin{solution}
\end{solution}

\bigskip

\begin{question}[4.]
\end{question}
\begin{solution}
A pair of dice is rolled until a sum of either 
$5$ or $7$ appears. We wish to compute the probability that a $5$
occurs first.
Let $E_n$ denote the event, where a $5$ occurs on the $n$th roll,
and no $5$ or $7$ occurs on the first $(n-1)$ roll.
The probability that we wish to compute can be written as
\eQb
\sum_{n=1}^{\infty} \mathbb{P}( E_n ),
\eQe
as each $E_n$ are pairwise disjoint events.
Let $F_n$ denote the event, where no $5$ or $7$ occurs on the $n$th roll,
and let $G_n$ denote the event, where $5$ occurs on the $n$th roll.
\eQb
\mathbb{P}(E_n ) &=& \mathbb{P}((\cup_{k=1}^{n-1}F_k) \cup G_n). 
\eQe
Now, with the independence assumption on each throw, we can factorize the RHS,
and obtain
\eQnb
\mathbb{P}(E_n ) &=& \mathbb{P}(G_n) \prod_{k=1}^{n-1}\mathbb{P}(F_{k}), 
\eQne
where $n-1 = 0$ case for the product term is defined to be $1$.
Assuming that the two dices are both fair dices, through a simple combinatorial argument(just count!),
we can obtain that
\eQb
\mathbb{P}(G_n) &=& \dfrac{4}{36} = \dfrac{1}{9}, \\
\mathbb{P}(F_n) &=& 1 - \dfrac{4}{36} - \dfrac{6}{36} = \dfrac{26}{36} = \dfrac{13}{18}. 
\eQe
Substituting the above equations into $(1)$, we have
\eQb
\mathbb{P}(E_n) &=& \dfrac{1}{9}\prod_{k=1}^{n-1} \dfrac{13}{18} \\
&=& \dfrac{1}{9} (\dfrac{13}{18})^{n-1}
\eQe
Consequently, we obtain
\eQb
\sum_{n=1}^{\infty}\mathbb{P}(E_n) &=& \sum_{n=1}^{\infty}
\dfrac{1}{9} (\dfrac{13}{18})^{n-1} \\
&=& \dfrac{1}{9} (\dfrac{1}{1 - \frac{13}{18}}) \\
&=& \dfrac{2}{5}.
\eQe
Therefore, the probability that a $5$ occurs first is $\dfrac{2}{5}$. $\qed$

\end{solution}

\bigskip

\begin{question}[5. $\limsup$ and $\liminf$ ]
\end{question}
\begin{solution}
\end{solution}

\bigskip

\begin{question}[6.]
\end{question}
\begin{solution}
\end{solution}

\bigskip



\end{document}
