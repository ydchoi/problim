\documentclass{article} % For LaTeX2e
\usepackage{nips14submit_e,times}
\usepackage{amsmath}
\usepackage{amsthm}
\usepackage{amssymb}
\usepackage{mathtools}
\usepackage{hyperref}
\usepackage{url}
\usepackage{algorithm}
\usepackage[noend]{algpseudocode}
%\documentstyle[nips14submit_09,times,art10]{article} % For LaTeX 2.09

\usepackage{graphicx}
\usepackage{caption}
\usepackage{subcaption}

\def\eQb#1\eQe{\begin{eqnarray*}#1\end{eqnarray*}}
\def\eQnb#1\eQne{\begin{eqnarray}#1\end{eqnarray}}
\providecommand{\e}[1]{\ensuremath{\times 10^{#1}}}
\providecommand{\pb}[0]{\pagebreak}


\def\Qb#1\Qe{\begin{question}#1\end{question}}
\def\Sb#1\Se{\begin{solution}#1\end{solution}}

\newenvironment{claim}[1]{\par\noindent\underline{Claim:}\space#1}{}
\newtheoremstyle{quest}{\topsep}{\topsep}{}{}{\bfseries}{}{ }{\thmname{#1}\thmnote{ #3}.}
\theoremstyle{quest}
\newtheorem*{definition}{Definition}
\newtheorem*{theorem}{Theorem}
\newtheorem*{lemma}{Lemma}
\newtheorem*{question}{Question}
\newtheorem*{preposition}{Preposition}
\newtheorem*{exercise}{Exercise}
\newtheorem*{challengeproblem}{Challenge Problem}
\newtheorem*{solution}{Solution}
\newtheorem*{remark}{Remark}
\usepackage{verbatimbox}
\usepackage{listings}
\title{Basic Probability: \\
Problem Set I}


\author{
Youngduck Choi \\
CILVR Lab \\
New York University\\
\texttt{yc1104@nyu.edu} \\
}


% The \author macro works with any number of authors. There are two commands
% used to separate the names and addresses of multiple authors: \And and \AND.
%
% Using \And between authors leaves it to \LaTeX{} to determine where to break
% the lines. Using \AND forces a linebreak at that point. So, if \LaTeX{}
% puts 3 of 4 authors names on the first line, and the last on the second
% line, try using \AND instead of \And before the third author name.

\newcommand{\fix}{\marginpar{FIX}}
\newcommand{\new}{\marginpar{NEW}}

\nipsfinalcopy % Uncomment for camera-ready version

\begin{document}


\maketitle

\begin{abstract}
This work contains a collection of solutions for selected problems 
of the Basic Probability course of Fall 2015.
\end{abstract}

\begin{question}[1.dd]
\end{question}
\begin{solution}
\textbf{(i)}
Let $X = \{ (a,b) | a \in \mathbb{Q}, b \in \mathbb{Q} \}$. In particular, $X = \mathbb{Q}
\times \mathbb{Q}$. Notice that by the Cantor diagonalization argument, there exists
a bijective map $\phi: \mathbb{Q} \times \mathbb{Q} \to \mathbb{N} \times \mathbb{N}$. Now,
consider a map $\psi: \mathbb{N} \times \mathbb{N} \to \mathbb{N}$ such that $\psi(m,n) = (m+n)^2 + n$.
With some algebra, we can show that if $\psi(m,n) = \psi(m',n')$, then $m = m'$ and $n = n'$. Hence,
$\psi$ is injective, and $\mathbb{N} \times \mathbb{N}$ is equipotent to $\psi(\mathbb{N} \times \mathbb{N})$,
which is a subset of the countable set $\mathbb{N}$. Therefore, we have shown that $\mathbb{N} \times \mathbb{N}$
is countable, and consequently $X$ is countable, due to the existence of the bijective map $\phi$.

\bigskip

\textbf{(ii)}
Let $X = \{ B((x,y),r) | x = y \}$. Consider the set $Y$ such that $Y = \{ B((x,y),1) | x = y \}$. $Y$ is in fact
equipotent with $\mathbb{R}$, as you can arbitrarily pick the center from $\mathbb{R}$ and that determines the
circle in $Y$. Hence, $Y$ is uncountable. Furthermore, notice that $Y \subset X$. Hence, $X$ is uncountable.

\bigskip

\textbf{(iii)}



\end{solution}

\bigskip

\begin{question}[2-(i). $\sigma$-field]
\end{question}
\begin{solution}
Let $\{ \mathcal{G}_{\lambda} \}_{\lambda \in \Lambda}$ be a collection of $\sigma$-fields
of the space $\Omega$.
We wish to show that $\cap_{\lambda \in \Lambda} \mathbb{G}_{\lambda}$ is a $\sigma$-field of $\Omega$.
As $\emptyset$, $\Omega \in \mathcal{G}$ for all $\lambda \in \Lambda$, we have that
\[
\emptyset, \>\> \Omega \in \cap_{\lambda \in \Lambda} \mathcal{G}_{\lambda},
\]
thereby satisfying one basic property of $\sigma$-field. It remains to show that 
a union of countable collection of subsets in 

\end{solution}

\bigskip

\begin{question}[4]
\end{question}
\begin{solution}
\end{solution}



\end{document}
